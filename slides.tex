\documentclass[brazil]{beamer}
\usepackage{beamerthemesplit}
\usepackage[brazilian]{babel}
\usepackage[utf8]{inputenc}
\usepackage{color}
\usepackage{xcolor}
\usepackage{graphicx}
\usepackage{float}
\usepackage{wrapfig}
\usepackage{amssymb}
\usepackage{amsmath}
\usepackage{fancybox}
\usepackage{ulem}
\usepackage{listings}
\usepackage{upquote}
\usetheme{JuanLesPins}

\title{
  Exergames and Beyond
}
\subtitle{
  Usando tecnologia para incentivar atividade física
}
\author{Fernando Omar Aluani (aluani@ime.usp.br)}

\begin{document}


\frame{
  \titlepage
}

\frame{\tableofcontents}
%
% explicar exergames
% explicar motivação, pq eh importante exergame (obesidade)
% explicar problemas atuais
% falar de algumas pesquisas (os 4 artigos principais)
% bibliografia
%
% 1) Exergames
%   A) O que são? Exemplos
%   B) Importância
%   C) Problemas atuais
%
% 2) Pesquisas Atuais
%   A) Interação no Escuro
%   B) Exergames de ação para crianças com paralisia cerebral
%   C) Temas de Design para Skateboarding
%   D) Entendendo Uso de Exergames ao Longo do Tempo
%
% 3) Conclusão
%   A) conclusão
%   B) bibliografia
%

%-------------------------------------
\section{Exergames}
%-------------------------------------
\frame{
  \begin{center}
  \LARGE 1. Exergames
  \end{center}
}
%-------------------------------------

%-------------------------------------
\subsection{O que são?}
%-------------------------------------
\begin{frame}
  \frametitle{O que são?}
  \vspace{-10pt}
  \begin{itemize}
    \pause
    \item Jogos digitais.
    \vspace{10pt}
    \item Envolvem esforço físico do usuário/jogador.
    \vspace{10pt}
  \end{itemize}

  Jogos físicos aperfeiçoados com componentes digitais e que envolvam esforço físico também contam.
\end{frame}
%-------------------------------------
\begin{frame}
  \frametitle{Alguns Exemplos}
  \vspace{-10pt}
  \begin{itemize}
    \pause
    \item Vários jogos do Wii, como Wii Sports e Wii Fit.
    \vspace{10pt}
    \item Vários jogos de XBox que usam o Kinect, como Dance Central.
    \vspace{10pt}
    \item Pokémon HeartGold e SoulSilver, com acessório externo.
    \vspace{10pt}
    \item Um bambôle ou corda de pular com sensores de rotação.
  \end{itemize}
\end{frame}

%-------------------------------------
\subsection{História e Importância}
%-------------------------------------
\begin{frame}
  \frametitle{História}
  \vspace{-10pt}
  \begin{itemize}
    \pause
    \item Já existem exergames fazem anos, desde década de 80/90.
    \vspace{10pt}
    \item Tiveram sucesso limitado no passado.
    \vspace{10pt}
    \item Na ultima década foi aumentando interesse e uso.
    \vspace{10pt}
    \item Tecnologias novas como Wii, Kinect e Smartphones facilitaram 
      desenvolvimento e adoção de exergames.
  \end{itemize}
\end{frame}
%-------------------------------------
\begin{frame}
  \frametitle{Importância}
  \vspace{-10pt}
  \begin{itemize}
    \pause
    \item Trazem exercício físico para uma atividade sedentária.
    \vspace{10pt}
    \item Falta de exercício físico leva à problemas de saúde.
    \vspace{10pt}
    \item Como obesidade, problemas cardíacos, entre outros.
    \vspace{10pt}
  \end{itemize}
  \pause
  Tais fatores tem levantado interesse da comunidade acadêmica.
\end{frame}

%-------------------------------------
\subsection{Problemas}
%-------------------------------------
\begin{frame}
  \frametitle{Problemas Atuais}
  \vspace{-10pt}
  \begin{itemize}
    \pause
    \item Atividade física de forma inadequada.
    \vspace{10pt}
    \item Problemas de acessabilidade.
    \vspace{10pt}
    \item Efeito platô.
  \end{itemize}
\end{frame}

%-------------------------------------
\section{Pesquisas Atuais}
%-------------------------------------
\frame{
  \begin{center}
  \LARGE 2. Pesquisas Atuais
  \end{center}
}

%-------------------------------------
\subsection{Interação no Escuro}
%-------------------------------------
\begin{frame}
  \frametitle{Interação no Escuro}
  \vspace{-10pt}
  Estudam uso e impacto social e emocional de um sistema de interação no escuro.
  
  \vspace{10pt}
  \pause
  Perceberam que no escuro as pessoas:
  \begin{itemize}
    \item usam a imaginação para ``ver'' o mundo.
    \vspace{10pt}
    \item se movimentam mais lentamente.
    \vspace{10pt}
    \item ficavam mais relaxadas e conscientes com seu corpo.
    \vspace{10pt}
    \item se preocupavam mais com o que sentiam do que com aparência visual.
    \vspace{10pt}
    \item alguns ficavam cuidadosos, outros livres.
  \end{itemize}
\end{frame}

%-------------------------------------
\subsection{Exergames de Ação para Crianças com Paralisia Cerebral}
%-------------------------------------
\begin{frame}
  \frametitle{Exergames de Ação para Crianças com Paralisia Cerebral}
  \vspace{-10pt}
  Estudam desenvolvimento de exergames de ritmo rápido para crianças com paralisia
  cerebral (PC), pois  
  \vspace{10pt}
  \pause
  \begin{itemize}
    \item elas também querem jogar jogos de ação com ritmo rápido.
    \vspace{10pt}
    \item querem usar exergame como fisioterapia, para evitar fadiga muscular.
    \vspace{10pt}
    \item diretrizes tradicionais de design de jogos para crianças com PC não permitem jogos rápidos.
  \end{itemize}
\end{frame}
%-------------------------------------
\begin{frame}
  \frametitle{Pontos Importantes no Design}
  \vspace{-10pt}
  Com o estudo, descobriram que os seguintes pontos, quando alterados, melhoravam 
  bastante o jogo para as crianças com PC:
  \vspace{10pt}
  \begin{itemize}
    \item Simplificar desenho e fluxo do mapa do jogo.
    \vspace{10pt}
    \item Reduzir consequências de erros.
    \vspace{10pt}
    \item Limitar ações possíveis.
    \vspace{10pt}
    \item Remover necessidade de posicionamento e mira precisa.
    \vspace{10pt}
    \item Deixar estado do jogo bem vísivel.
    \vspace{10pt}
    \item Balancear o jogo para abilidades físicas diferentes.
  \end{itemize}
\end{frame}


%-------------------------------------
\subsection{Temas de Design para Skateboarding}
%-------------------------------------
\begin{frame}
  \frametitle{Temas de Design para Skateboarding}
  \vspace{-10pt}
  Estudam como desenhar tecnologias interativas para esportes que focam na experiência de 
  realizar truques, como o skateboarding.
  
  \vspace{10pt}
  \pause
  Formaram os seguintes temas para design:
  \begin{itemize}
    \item Localização de feedback em relação ao corpo do skatista.
    \vspace{10pt}
    \item Timing de feedback em relação com os picos emocionais depois de truques.
    \vspace{10pt}
    \item Aspectos do truque enfatizados pelo feedback.
    \vspace{10pt}
    \item Estética apropriada do feedback
  \end{itemize}
\end{frame}

%-------------------------------------
\subsection{Entendendo Uso de Exergames ao Longo do Tempo}
%-------------------------------------
\begin{frame}
  \frametitle{Entendendo Uso de Exergames ao Longo do Tempo}
  \vspace{-10pt}
  Estudam motivação e comportamento dos usuários de exergames ao longo do tempo,
  e como melhorar esses aspectos.
  
  \vspace{10pt}
  \pause
  Notaram como a teoria de \textit{self-efficacy} é importante para determinar
  motivação e comportamento de um jogador, e levando isso em consideração pode
  ajudar a manter a motivação do usuário ao longo do tempo.
\end{frame}

%-------------------------------------
\section{Conclusão}
%-------------------------------------
\frame{
  \begin{center}
  \LARGE 3. Conclusão
  \end{center}
}
%-------------------------------------
\begin{frame}
  \frametitle{Bibliografia}
  \begin{itemize}
    \footnotesize
    \item[1]
      Linden Vongsathorn, Kenton O'Hara, and Helena M. Mentis. 2013. Bodily interaction in the dark. 
    \vspace{1em}
    \item[2]
      Hamilton A. Hernandez, Zi Ye, T.C. Nicholas Graham, Darcy Fehlings, and Lauren Switzer. 2013. 
      Designing action-based exergames for children with cerebral palsy. 
    \vspace{1em}
    \item[3]
      Sebastiaan Pijnappel and Florian Mueller. 2013. 4 design themes for skateboarding. 
    \vspace{1em}
    \item[4]
      Andrew Macvean and Judy Robertson. 2013. Understanding exergame users' physical activity, motivation and behavior over time. 
  \end{itemize}
  In Proceedings of the SIGCHI Conference on Human Factors in Computing Systems (CHI '13). ACM, New York, NY, USA.
\end{frame}

\end{document}
